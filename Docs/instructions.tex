\documentclass[10pt,a4paper]{article}

\usepackage[utf8]{inputenc}
\usepackage[english]{babel}
\usepackage[english]{isodate}
\usepackage[parfill]{parskip}
\usepackage{natbib}
\usepackage{url}

% Journals

\newcommand{\adv}{    {\it Adv. Spa. Res.}}
\newcommand{\annG}{   {\it Annales Geophysicae}}
\newcommand{\aap}{    {\it Astron. Astrophys.}}
\newcommand{\aaps}{   {\it Astron. Astrophys. Suppl.}}
\newcommand{\aapr}{   {\it Astron. Astrophys. Rev.}}
\newcommand{\ag}{     {\it Ann. Geophys.}}
\newcommand{\aj}{     {\it Astronom. J.}}
\newcommand{\apj}{    {\it Astrophys. J.}}
\newcommand{\apjl}{   {\it Astrophys. J. Lett.}}
\newcommand{\aplett} { {\it Astrophys. Lett.}}
\newcommand{\apjs}{   {\it Astrophys. J. Suppl. Ser.}}
\newcommand{\apss}{   {\it Astrophys. Spa. Sci.}}
\newcommand{\cjaa}{   {\it Chinese J. Astron. Astrophys.}}
\newcommand{\gafd}{   {\it Geophys. Astrophys. Fluid Dyn.}}
\newcommand{\grl}{    {\it Geophys. Res. Lett.}}
\newcommand{\ijga}{   {\it Int. J. Geomag. Aeron.}}
\newcommand{\jastp}{  {\it J. Atmos. Sol. Terr. Phys.}}
\newcommand{\jgr}{    {\it J. Geophys. Res.}}
\newcommand{\mnras}{  {\it Mon. Not. Roy. Astron. Soc.}}
\newcommand{\nat}{    {\it Nature}}
\newcommand{\pasp}{   {\it Pub. Astron. Soc. Pac.}}
\newcommand{\pasj}{   {\it Pub. Astron. Soc. Japan}}
\newcommand{\pre}{    {\it Phys. Rev. E}}
\newcommand{\solphys}{{\it Solar Phys.}}
\newcommand{\sovast}{ {\it Sov. Astronom.}}
\newcommand{\ssr}{    {\it Space Sci. Rev.}}
\newcommand{\araa}{    {\it	ARA\&A; KNUDSEN}}
\newcommand{\anach}{    {\it	Astronomische Nachrichten}}


\begin{document}

\begin{titlepage}
   \vspace*{\stretch{1.0}}
   \begin{center}
      \Large\textbf{Instruction Manual for TAMCMC-C++}\\
      \textbf{Version 1.3.1} \\
      \large\textit{Othman Benomar}
   \end{center}
   \vspace*{\stretch{2.0}}
\end{titlepage}

\pagenumbering{Roman}
\tableofcontents
\newpage
%\listoffigures
%\newpage
%\listoftables
%\newpage
\pagenumbering{arabic}


\section{Purpose of TAMCMC-C++}

	TAMCMC stands for Tempered Adaptive Markov Chain Monte Carlo and is a multipurpose code designed to analyse asteroseismic data. As the extension suggests, the code is written in C++, but is based on my PhD work\footnote{\url{http://www.theses.fr/2010PA112309}}, but is also described in \cite{Benomar2009}. The algorithm scheme is taken from \cite{atchade2006}. It is designed to
	\begin{itemize}
		\item Automatically adjust the covariance matrix of the proposal law in order to optimally sample a statistical criteria with a MCMC algorithm.
		\item Use the local gradient to enhance the exploration properties of the algorithm. This is this 'Langevin' part of the algorithm that lead to the algorithm name: MALA (Metropolis-Hasting-Langevin Algorithm).   
	\end{itemize}
The current version of the code does not implement the 'Langevin' scheme, because the derivative is not known a priori when analysing seismic data. However, this might be implemented in the future.

This version of the code is designed to perform power spectrum fitting of Main-sequence solar-like stars. It comes with a suite of models that allows to perform global fits of individual pulsations. Although this was not tested thoroughly, It has a also a module to handle the fit of mode envelopes, considering a Gaussian envelope over a noise background. 


\section{Basic structure of the code and efficiency}
The code is built so that most user do not require any coding skills for analysing basic asteroseismic data. Furthermore, if it is needed to add/modify models, this is fairly easy due to the modular structure of the code: The core computational part is independent from the model/priors/data/statistical criteria that should be used. Adding or Modifying those requires only to edit few files,
\begin{itemize}
	\item[-]  \textbf{\path{./tamcmc/sources/models.cpp}}  that contains all the models that must computed
	\item[-] \textbf{\path{./tamcmc/sources/config.cpp} }   that contains the list of models, priors and statistics that are accepted.
	\item[-] \textbf{\path{./tamcmc/sources/model_def.cpp} } that specify which function and what arguments must be used depending on the chosen model
\end{itemize}

TAMCMC-C++ is build such that it is highly efficient in sampling the pdf of a distribution. Parallelisation of the parallel chains is possible by OpenMP. Hence, when analysing typically $\approx 10^5$ data points\footnote{Typical number of the longest {\it Kepler} observations.} the typical time to compute 1 Million samples is of a couple of days on standard quad core CPU.

\section{General setup}

\section{Model setup}

\section{External tools}

\bibliographystyle{humannat}
\bibliography{bib_othman}

\end{document}